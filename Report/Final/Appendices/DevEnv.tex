%!TEX root = ../Main.tex
% Development Environement
\chapter{Development Environment} % (fold)
\label{apdx:development_environment}


\section{FPGA design cycle} % (fold)
\label{apdx:fpga_design_cycle}
	As mentioned in the report, Synplify Premier was used to synthesise the SystemVerilog code.  Then, the ISE Webpack (including the Floorplanner software) was used to assign ports to physical pins, and generate programming files.  Finally, urJTAG was used to actually program the device over JTAG.

	In order to use Xilinx design tools with Synplify, the correct environment variable must be set.  Under the Tools menu, select %TODO!!!

	In addition, the implementation options for Synplify must be correctly set -- in particular the correct part must be selected, and appropriate timing constraints entered.  The onboard 50MHz clock needs one such constraint.

	Once this is set up, the design process roughly follows this sequence:
	\begin{enumerate}
		\item Write SystemVerilog code.
		\item Run implementation; fix errors until it synthesises correctly.
		\item Open project with Xilinx design tools (via Tools menu).
		\item Either manually edit the constraints file and add net to port mappings, or use the Floorplanner (post-synthesis) tool to do this.
		\item Run implementation.  If mapping fails, your design may be too big to fit on the device.
		\item Generate .bit programming file.
		\item Open the `Impact' software (which comes with the ISE Webpack).
		\item Create a new project, and select the .bit file created in the previous step.
		\item From the Output menu, select Create new SVF file.
		\item Perform whichever actions you want from the list on the left -- programming the flash takes a long time, so if you're still in the testing stages, only run ``Program FPGA''.
		\item End the SVF output (from the same menu as before).  This step is very important!
		\item Open urJTAG and connect to the programming cable.
		\item Use `svf programming-file.svf' to transfer your design to La Papessa!
	\end{enumerate}

	As this process is fairly long, it may be useful to speed some parts up.  In particular, instead of repeatedly assigning pins with Floorplanner, it's easier to copy the constraints file once everything is assigned correctly.  Then when Webpack is opened again, and the constraints are automatically overwritten, you can simply copy in the correct pin assignments.  The constraints used for this project are available in the file archive, in the DevEnv folder.
% sectapdx fpga_design_cycle (end)



\section{LTIB usage} % (fold)
\label{apdx:ltib_usage}

	\subsection{Cross compiling} % (fold)
	\label{apdx:cross_compiling}
		To compile code that will work on L'Imperatrice, you need to use the binaries installed by LTIB.  They are normally stored in:

		\texttt{/opt/freescale/usr/local/gcc-x-glibc-x-x/arm-none-gnueabi/\\*
		arm-none-linux-gnueabi/bin/}
		  
		Running the following command will add this to your path (\$ is your prompt):

		\texttt{\$ PATH=/opt/freescale/usr/local/gcc-4.1.2-glibc-2.5-nptl-3/\\*
		arm-none-linux-gnueabi/arm-none-linux-gnueabi/bin/:\$PATH}

		Then running `gcc' will automatically use the arm version.
	% subsectapdx cross_compiling (end)


	\subsection{GPIO and UART} % (fold)
	\label{apdx:gpio_and_uart}
		Both GPIO and UART are optional components that must be enabled in the LTIB configuration.  
			%TODO!!!
	% subsectapdx gpio_and_uart (end)


	\subsection{Compiling FFTW for LTIB} % (fold)
	\label{apdx:compiling_fftw_for_ltib}
		%Read: http://ltib.org/documentation-LtibFaq
		In many cases you may want to build a library or package to install on L'Imperatrice.  For example, you may want to perform real time Fast Fourier Transforms using the FFTW library, which has native support for ARMv5 devices.  The general procedure, as described in the FAQ at \href{http://ltib.org/documentation-LtibFaq}{http://ltib.org/documentation-LtibFaq}, is:
		\begin{enumerate}
			\item Prepare the source files
			\item Build and install the package to the rootfs using LTIB
			\item Add -I and -L flags to gcc if you want to cross compile with the libraries you've just installed.
		\end{enumerate}

		The FFTW source can be downloaded from \href{http://www.fftw.org/download.html}{http://www.fftw.org/download.html}.  In order to add it to LTIB, a custom RPM spec file was created, available in the project file archive in the DevEnv folder.  Then the ltib binary is used to correctly configure and build FFTW from source, and add it to the list of installed packages.  This allows FFTW to be selected in the `Packages list' in the LTIB configuration menu, and thus installed in the rootfs.

		When compiling a C program that uses FFTW, the relevant library and include paths must be added.  In this project, a Makefile was used to automate the build process.
	% subsection compiling_fftw_for_ltib (end)
% section ltib_usage (end)


% chapter development_environment (end)
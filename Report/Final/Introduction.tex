%!TEX root = Main.tex
% Introduction: words
\chapter{Introduction} % (fold)
\label{cha:introduction}

% Introduce the project, talk about the planned and final goals, my contribution, what I learnt, and a short conclusion.
% Make it clear that it's a POC system

\section{Goals} % (fold)
\label{sec:goals}
	At the highest level, the primary goal of this project is to implement part of a modern HMM based speech recognition system, with the constraint that it must be done on embedded hardware.  In pursuing this goal, the aim is to achieve several other goals that will be beneficial to the author and to the University of Southampton.

	\subsection{Speech Recognition} % (fold)
	\label{sec:speech_recognition}
	%It is hoped that the system may be used as a basis for future research into the subject, and
		The 
	% subsection speech_recognition (end)

	\subsection{Theoretical understanding} % (fold)
	\label{sec:theoretical_understanding}
		A major goal of the project is to develop a higher level of understanding of the algorithms used in speech recognition, and to get experience designing a large-scale embedded application.  This complements the interests of the author and the subjects being studied, in particular, Intelligent Algorithms and Digital Systems Design.
	% subsection theoretical_understanding (end)

	\subsection{The Micro Arcana} % (fold)
	\label{sec:the_micro_arcana}
		In terms of hardware, one of the project goals is to further development of the Micro Arcana family of boards, and provide a valuable use case example as described in the Motivation section.  Part of the project is setting up and configuring these two boards, so that they may be easily picked up by undergraduates.  In addition, the aim is to build the entire system on these two boards, making it a self-contained embedded design.
	% subsection the_micro_arcana (end)

% section goals (end)

\section{Motivation} % (fold)
\label{sec:motivation}
	Speech recognition is an interesting computational problem, for which there is no fool-proof solution at this time.  Recently the industry for embedded devices and small-scale digital systems has expanded greatly, but in general these devices do not have the power or speed to run speech recognition.  Field Programmable Gate Arrays (FPGAs) may present a way of increasing the capability of such systems, as they are able to perform calculations much faster than traditional microprocessors.  

	The Micro Arcana is a new hardware platform aimed at undergraduate students, being developed by Dr Steve Gunn.  As they are still under development, they are very untested, and very little documentation exists.  In order to improve their reception by students, it would greatly help to have proven use cases and examples of how these boards may be used individually and together.  Using a larger FPGA (such as an Altera DE board) was considered during the planning stages of this project, but it was decided that part of the challenge was to develop and use the Micro Arcana.
% section motivation (end)

\section{Contributions} % (fold)
\label{sec:contributions}
	The project implements one part of a modern speech recognition system, using two development boards from the Micro Arcana family.  It is designed to be a proof of concept exercise, in order to explore the capabilities of the boards, and expand the author's knowledge of the relevant systems.  Specifically, the project required substantial research into HMM based speech recognition systems, embedded Linux, and digital design.  The resulting system, described in detail in Chapter~\ref{cha:system_design}, uses an FPGA to perform the most computationally expensive part of HMM based recognisers -- scoring the states of each HMM model for a given input vector.  Essentially, the ARM based ``L'Imperatrice'' is used as the application controller, and is connected to the FPGA based ``La Papessa'' board which performs the CPU intensive mathematical calculations.  Given an observation vector, the FPGA will process it and send back scores for each state in the speech model.  The other main processes in a speech recogniser, such as pre-processing and decoding, are tasks that are well suited to software implementation.
% section contributions (end)


% chapter introduction (end)

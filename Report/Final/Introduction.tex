%!TEX root = Main.tex
% Introduction: words
\chapter{Introduction} % (fold)
\label{cha:introduction}

\section{Goals} % (fold)
\label{sec:goals}
	At the highest level, the primary goal of this project is to implement part of a modern Hidden Markov Model (HMM) \nomenclature{HMM}{Hidden Markov Model} based speech recognition system, with the constraint that it must be done on embedded hardware.  This and other important goals are outlined here.

	\subsection{Speech Recognition} % (fold)
	\label{sec:speech_recognition}
		Most modern HMM based speech recognition systems are extremely complex, and can take years to design and optimise for a particular implementation.  The goal of this project is not to implement such a system, but rather to explore the possibilities of what may be achieved with a low power applications processor and a relatively small FPGA (Field Programmable Gate Array \nomenclature{FPGA}{Field Programmable Gate Array}.  In particular, the aim is to use the applications processor (running Linux) to perform pre-processing of speech data, and use the FPGA to perform mathematical manipulations of the observations.  Part of this goal is to evaluate the performance, in terms of calculation speed, of the FPGA relative to a conventional processor.  Finally, it is hoped that the system developed here may be later used as a basis for future research into the subject.
	% subsection speech_recognition (end)

	\subsection{The Micro Arcana} % (fold)
	\label{sec:the_micro_arcana}
		The Micro Arcana is a new hardware platform aimed at undergraduate students, currently under development by Professor Steve Gunn.
		In terms of hardware, one of the project goals is to further development of the Micro Arcana family of boards, and provide a valuable example of how they may be usefully combined.  The aim is to build the entire speech recognition system on two of the Micro Arcana boards, making it a self-contained embedded design.  In addition, part of the project is setting up and configuring these two boards, so that they may be easily picked up by undergraduates.
	% subsection the_micro_arcana (end)

	\subsection{Theoretical understanding} % (fold)
	\label{sec:theoretical_understanding}
		An important goal of the project is to develop a higher level understanding of the algorithms used in speech recognition, and to get experience designing a large-scale embedded application.  This project encompasses a very wide range of subjects, including intelligent algorithms, digital systems design, embedded processing, and hardware design.
		% TODO develop
	% subsection theoretical_understanding (end)

% section goals (end)


\section{Motivation} % (fold)
\label{sec:motivation}
	Speech recognition is an interesting computational problem, for which there is no fool-proof solution at this time.  Recently the industry for embedded devices and small-scale digital systems has expanded greatly, but in general these devices do not have the power or speed to run speech recognition.  FPGAs may present a way of increasing the capability of such systems, as they are able to perform calculations much faster than traditional microprocessors.  The versatility of embedded systems, combined with the challenges of a complex system such as speech recognition, makes this an appealing area to explore.

	As hardware platforms go, most of the Micro Arcana is still very new and untested, as it is still under development.  In addition, there are not many examples of how they may be used, and very little documentation.  In order to improve their reception by students, it would greatly help to have proven use cases and examples of how these boards may be used individually and together.  Using a larger FPGA (such as an Altera DE board) was considered during the planning stage of this project, but it was decided that it would be more beneficial and interesting to develop and use the Micro Arcana.
% section motivation (end)


\section{Results and Personal Contribution} % (fold)
\label{sec:contributions}
	The project implements two parts of a modern speech recognition system, using two development boards from the Micro Arcana family.  The implementation, described in detail in Chapter~\ref{cha:system_design}, uses an FPGA to perform the most computationally expensive part of HMM based recognisers -- scoring the states of each HMM model for a given input vector.  Essentially, the ARM Linux based ``L'Imperatrice'' is used as the application controller, and is connected to the FPGA based ``La Papessa'' board.  The processor reads Microsoft WAV format speech files, performs the necessary pre-processing and sends observation vectors to the FPGA.  Given an observation, the FPGA processes it and sends back scores for each state in the speech model, which represent the probability of that state emitting the observation.  Given these scores, the next step for a speech recogniser would be to perform Viterbi decoding (possibly using token-passing or a similar algorithm) in order to find the most probable sequence of HMMs, and thus eventually find the most probable word or phoneme sequence spoken.

	Accomplishing this required substantial research into HMM based speech recognition algorithms, embedded Linux, and digital design.  The two Micro Arcana boards are very new, and some vital parts on them were completely untested before this project.  Although there is significant research into the use of FPGAs for speech recognition, most cases use large FPGAs, and often a relatively fast PC to perform pre-processing.  This project is designed to be a proof of concept exercise -- it explores the capabilities of the boards, and provides an example of how they may be used together.  Furthermore, the results are satisfactory, and point towards an FPGA being an appropriate platform for performing these calculations.
% section contributions (end)


% chapter introduction (end)

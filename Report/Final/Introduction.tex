%!TEX root = Main.tex
% Introduction: words
\chapter{Introduction} % (fold)
\label{cha:introduction}

% Introduce the project, talk about the planned and final goals, my contribution, what I learnt, and a short conclusion.
% Make it clear that it's a POC system

\section{Goals} % (fold)
\label{sec:goals}
	At the highest level, the primary goal of this project is to implement part of a modern HMM based speech recognition system, with the constraint that it must be done on embedded hardware.  In pursuing this goal, the aim is to achieve several other goals that will be beneficial to the author and to the University of Southampton.

	\subsection{Speech Recognition} % (fold)
	\label{sec:speech_recognition}
		A complete HMM based speech recognition system is extremely complex, and can take years to design and optimise for a particular implementation.  The goal of this project is not to implement one such system, but rather to explore the possibilities of what may be achieved with a low power applications processor and a relatively small FPGA.  In particular, the goal is to use the applications processor (running Linux) to perform pre-processing of speech data, and use the FPGA to perform mathematical manipulations of the observations.  It is hoped that the system developed here may be later used as a basis for future research into the subject.
	% subsection speech_recognition (end)

	\subsection{The Micro Arcana} % (fold)
	\label{sec:the_micro_arcana}
		The Micro Arcana is a new hardware platform aimed at undergraduate students, currently under development by Dr Steve Gunn.  
		In terms of hardware, one of the project goals is to further development of the Micro Arcana family of boards, and provide a valuable use case example as described in the Motivation section.  Part of the project is setting up and configuring these two boards, so that they may be easily picked up by undergraduates.  In addition, the aim is to build the entire system on these two boards, making it a self-contained embedded design and showing how they may be usefully combined.
	% subsection the_micro_arcana (end)

	\subsection{Theoretical understanding} % (fold)
	\label{sec:theoretical_understanding}
		A major goal of the project is to develop a higher level of understanding of the algorithms used in speech recognition, and to get experience designing a large-scale embedded application.  This complements the interests of the author and the subjects being studied, in particular, Intelligent Algorithms and Digital Systems Design.
		% TODO develop
	% subsection theoretical_understanding (end)

% section goals (end)

\section{Motivation} % (fold)
\label{sec:motivation}
	Speech recognition is an interesting computational problem, for which there is no fool-proof solution at this time.  Recently the industry for embedded devices and small-scale digital systems has expanded greatly, but in general these devices do not have the power or speed to run speech recognition.  Field Programmable Gate Arrays (FPGAs) may present a way of increasing the capability of such systems, as they are able to perform calculations much faster than traditional microprocessors.  The author's personal interest in embedded systems, combined with the challenges of a complex system such as speech recognition, makes this an appealing area to explore.

	As hardware platforms go, most of the Micro Arcana is still very new and untested, as it is still under development.  In addition, there are not many examples of how they may be used, and very little documentation.  In order to improve their reception by students, it would greatly help to have proven use cases and examples of how these boards may be used individually and together.  Using a larger FPGA (such as an Altera DE board) was considered during the planning stages of this project, but it was decided that part of the challenge was to develop and use the Micro Arcana.
% section motivation (end)

\section{Contributions} % (fold)
\label{sec:contributions}
	The project implements one part of a modern speech recognition system, using two development boards from the Micro Arcana family.  It is designed to be a proof of concept exercise, in order to explore the capabilities of the boards, and expand the author's knowledge of the relevant systems.  Specifically, the project required substantial research into HMM based speech recognition systems, embedded Linux, and digital design.  The resulting system, described in detail in Chapter~\ref{cha:system_design}, uses an FPGA to perform the most computationally expensive part of HMM based recognisers -- scoring the states of each HMM model for a given input vector.  Essentially, the ARM based ``L'Imperatrice'' is used as the application controller, and is connected to the FPGA based ``La Papessa'' board which performs the CPU intensive mathematical calculations.  The embedded Linux processor reads WAV speech files, performs necessary pre-processing and sends the observation vectors to the FPGA.  Given this vector, the FPGA will process it and send back scores for each state in the speech model, which represent the probability of that state creating the observation.  Given these scores, the next step for a speech recogniser would be to perform Viterbi decoding in order to find the most probable sequence of HMMs, and thus eventually find the most probable word or phoneme sequence spoken.

	% TODO
	% Why is this useful?
	% What did I have to learn to accomplish this? 
% section contributions (end)


% chapter introduction (end)

%!TEX root = Main.tex
% Project management: words
\chapter{Project Analysis} % (fold)
\label{cha:project_management}

\section{Analysis of Solution} % (fold)
\label{sec:analysis_of_solution}
% Move to testing / conclusions ??

	\subsection{The FPGA} % (fold)
	\label{sub:analysis_the_fpga}
		The FPGA used was very small, and the full required design could not fit on it.  In particular, the size of the model had to be reduced, so that each Senone had fewer than 25 components, and not all 7000 senones were processed.  This is mainly due to the small amount of onboard RAM on the FPGA, so the design would greatly benefit from having more RAM available.
	% subsection the_fpga (end)

	% The fixed point accuracy? 16 bit?

	% Evaluate the speed - how long did each cycle take.  How much time was added by adding another component? or another senone? Extrapolate?

	% The communication method (UART) is extremely slow and is the weakest point of the system.  It was used primarily for the ease of implementation, and because at this stage real-time operations are not required.  However, for this system to be realistically useful, a better communication method needs to be developed -- either a faster serial bus or some form of parallel connection.
	% The baud rate used, 115200, is the fastest transfer speed usually supported by UART, but is still extremely slow compared to the calculation speed.  

	\subsection{The processor} % (fold)
	\label{sub:analysis_the_processor}
		% Weak points
		% The MFCC calculation has several problems.  The library used is slow and inefficient, and doesn't match the results given by HTK.
		% Do real sampling, not just reading wav files

	% subsection the_processor (end)

% section analysis_of_solution (end)


\section{Deviations From Initial Goals} % (fold)
\label{sec:deviations_from_initial_goals}

% section deviations_from_initial_goals (end)


\section{Time Management} % (fold)
\label{sec:time_management}

% section time_management (end)

% chapter project_management (end)
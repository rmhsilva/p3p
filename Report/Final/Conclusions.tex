%!TEX root = Main.tex
% Conclusions: words
\chapter{Conclusions} % (fold)
\label{cha:conclusions_and_future_work}

% Conclude....
% Re-state what was accomplished
% Personal gains - learnt loads, got to be the first doing something large with micro arcana, 
% Usefulness of result - someone interested in speech recognition, or 

This report presented a proof of concept system for performing speech recognition related operations on embedded hardware.  The system was built on two boards from the Micro Arcana family of development boards, and performed speech pre-processing and Gaussian distance calculations.  A background of the relevant speech recognition theory was given, along with examples of various other techniques and approaches.  Descriptions and analyses of the completed design were presented, along with information on how it was tested.  In addition to the hardware based system, a multi-purpose software toolkit was built that greatly helped during the design and testing stages of the project.

The principle goal, as described in Section~\ref{sec:goals}, was to design and implement part of an HMM based speech recognition system in embedded hardware, in order to evaluate and learn about such a system.  It was not intended to be used as a real recogniser, but rather to broaden the author's knowledge and experience, and provide a platform on which embedded speech recognition may be researched further.  The conclusions are given below.


\section{Personal Gains} % (fold)
\label{sec:personal_gains}
	As the author is still exploring the vast world of electronic engineering, one of the attractions of this project was its diversity and complexity.  The project involved several different areas of electronics, including digital systems design, complex algorithms and optimisation, embedded Linux programming, and hardware testing.  Completing this project, with all the different activities involved, was personally satisfying, especially as it involved working with hardware that had never been used in this manner before.
	% I gained lots of experience with complex algorithms, digital design,
	% Even though I did not implement the more complex algorithms, it was interesting learning about them, and how they work.
	% This is the largest digital system I have ever built, and I have never interfaced an FPGA with a C program.  I have never used an embedded Linux board that was so undeveloped, and I have never
% section personal_gains (end)


\section{Usefulness of Results} % (fold)
\label{sec:usefulness}
	Due to size limitations of the FPGA used, it was not possible to implement the full intended system.  However, it was still possible to estimate the relationship between the system speed and model size, which makes it possible to judge the usefulness of an FPGA in this case.  

	Overall, the design is a useful example of how two of the Micro Arcana family may be connected and used in a practical setting.  Some of the project's solutions, to difficulties encountered, may definitely be useful in other projects that use these boards.  
% section usefulness (end)


\section{Future Work} % (fold)
\label{sec:future_work}
	There are many possibilities for developing this project futher -- either improving the system already built, or implementing other speech recognition tasks such as decoding.  With the system already built, there are a few areas that may be interesting to invstigate and develop futher.

	Computing Mel Frequency Cepstral Coefficients were computed using an existing library for the purposes of this project.  However, this library is extremely slow, and definitely may be optimised.  In addition, it would be very beneficial if the pre-processing generated the same MFCCs as those generated by the HTK.  This would require an in-depth analysis of the HTK system in order to determine the exact sequence of operations that are performed.

	The communications method definitely needs futher work before it can be considered realistically usable.  A better system would need to be far faster, and possibly include features such as error-checking, handshaking, and control sequences.  Designing a better communications method between two of the Micro Arcana boards would be an interesting project, due to the size and speed contraints of the family.

	This project only implemented the first stages of a speech recognition, and did not at all focus on tasks such as decoding or language modelling.  However, it provides an environment where these systems may be implemented and used.  There is huge potential for futher work that could focus on any one of these areas.
% section future_work (end)


% chapter conclusions_and_future_work (end)
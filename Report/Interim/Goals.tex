%!TEX root = main.tex
% 243 words
\chapter{Project Goals} % (fold)
\label{cha:goals}

At the highest level, the goal of this project is to develop a speech recognition system that will run on embedded hardware.  In pursuing this goal, the aim is to achieve several other goals that will be beneficial to the author and to the University of Southampton.

\section{Speech Recognition} % (fold)
\label{sec:speech_recognition}
The speech recognition system proposed aims to be run using an ARM chip doing pre-processing and an FPGA doing statistical decoding.  Ideally, it will be capable of performing HMM based speaker-independent speech recognition.  This, however, is very ambitious, and so initially the project will focus on getting the most basic forms of the relevant algorithms running in software.  Once this is complete, the system will be built into hardware, and extended to be more complete.
% section speech_recognition (end)

\section{The Micro Arcana} % (fold)
\label{sec:the_micro_arcana}
In terms of hardware, one of the project goals is to further development of the Micro Arcana family of boards, currently under development by Dr Steve Gunn.  
The speech recognition system proposed will make use of two of the boards -- the ``L'Imperatrice'' ARM based mini-computer, and the ``La Papessa'' FPGA board.  Part of the project is setting up and configuring these two boards, and having a finished system will help demonstrate their capabilities.
% section the_micro_arcana (end)

\section{Theoretical understanding} % (fold)
\label{sec:theoretical_understanding}
A major goal of the project is to develop a higher level of understanding of the algorithms used in speech recognition, and to get experience designing a large-scale embedded application.  This complements the intrests of the author and the subjects being studied, in particular, Intelligent Algorithms and Digital Systems Design.
% section theoretical_understanding (end)


% chapter goals (end)